% Created 2024-04-08 Mon 10:36
% Intended LaTeX compiler: pdflatex
\documentclass[presentation]{beamer}
\usepackage[utf8]{inputenc}
\usepackage[T1]{fontenc}
\usepackage{graphicx}
\usepackage{longtable}
\usepackage{wrapfig}
\usepackage{rotating}
\usepackage[normalem]{ulem}
\usepackage{amsmath}
\usepackage{amssymb}
\usepackage{capt-of}
\usepackage{hyperref}
\mode<beamer>{\usetheme{Madrid}}
\definecolor{SUred}{rgb}{0.59375, 0, 0.17969} % SU red (primary)
\definecolor{SUblue}{rgb}{0, 0.17578, 0.38281} % SU blue (secondary)
\setbeamercolor{palette primary}{bg=SUred,fg=white}
\setbeamercolor{palette secondary}{bg=SUblue,fg=white}
\setbeamercolor{palette tertiary}{bg=SUblue,fg=white}
\setbeamercolor{palette quaternary}{bg=SUblue,fg=white}
\setbeamercolor{structure}{fg=SUblue} % itemize, enumerate, etc
\setbeamercolor{section in toc}{fg=SUblue} % TOC sections
% Override palette coloring with secondary
\setbeamercolor{subsection in head/foot}{bg=SUblue,fg=white}
\setbeamercolor{date in head/foot}{bg=SUblue,fg=white}
\institute[SU]{Shenandoah University}
\titlegraphic{\includegraphics[width=0.5\textwidth]{\string~/Documents/suLogo/suLogo.pdf}}
\newcommand{\R}{\mathbb{R}}
\usetheme{default}
\author{Chase Mathison\thanks{cmathiso@su.edu}}
\date{8 April 2024}
\title{Infinite Series}
\hypersetup{
 pdfauthor={Chase Mathison},
 pdftitle={Infinite Series},
 pdfkeywords={},
 pdfsubject={},
 pdfcreator={Emacs 29.1 (Org mode 9.6.7)}, 
 pdflang={English}}
\begin{document}

\maketitle

\section{Announcements}
\label{sec:orgbf3ff1c}
\begin{frame}[label={sec:org74d12f0}]{Announcements}
\begin{enumerate}
\item Homework/project.
\end{enumerate}
\end{frame}

\section{Lecture}
\label{sec:org9303e8d}
\begin{frame}[label={sec:org77232fa}]{Example}
Let
\[
S_k = 1 + \frac{1}{2} + \ldots + \frac{1}{2^k}\]
Let's try to find
\begin{enumerate}
\item A ``nicer'' way to write \(S_k\) and
\item \(\lim_{k\rightarrow \infty} S_k\)
\end{enumerate}
\vspace{10in}
\end{frame}

\begin{frame}[label={sec:orgdcee242}]{Example}
\end{frame}

\begin{frame}[label={sec:org4663060}]{Last time}
In the last example, we found the limit
of a sequence defined by
\[
S_k = \sum\limits_{n=1}^k \left( \frac{1}{2} \right)^n\]

When taking the limit of a sequence like this, it might make sense to write something
like
\[
\sum\limits_{n=1}^{\infty} \left( \frac{1}{2} \right)^n\]

Today we're going to start with things that look like
\[
\sum\limits_{n=1}^{\infty} a_n\]
and try to define what this means.  The object above is called an \uline{\hspace*{1in}}.
\end{frame}

\begin{frame}[label={sec:org1e359d1}]{Infinite Series Definition}
An infinite series is an expression of the form
\[
\sum\limits_{n=1}^{\infty} a_n = a_1 + a_2 + a_3 + \ldots \]
For each positive integer \(k\), the sum
\[
S_k = \sum\limits_{n=1}^k a_n\]
is called the \uline{\hspace*{2in}} of the series.  These partial sums form a
sequence \(\left\{ S_k \right\}\).  If the sequence of partial sums converges
to a real number \(S\), we say the infinite series \uline{\hspace*{1in}} to \(S\) and write
\[
\sum\limits_{n=1}^{\infty} a_n = S.\]
If the sequence of partial sums diverges, we say the infinite series \uline{\hspace*{1in}}.
\end{frame}

\begin{frame}[label={sec:org8b6fc5a}]{Example}
Find the sequence of partial sums to evaluate the following series:
\begin{enumerate}
\item \(\sum\limits_{n=1}^{\infty} \left( \frac{1}{3} \right)^n\)
\item \(\sum\limits_{n=0}^{\infty} \left( -1 \right)^n\)
\item \(\sum\limits_{n=1}^{\infty} \ln \left( \frac{n+1}{n} \right)\)
\item \(\sum\limits_{n=1}^{\infty} \frac{1}{n^2+n}\)
\end{enumerate}
\vspace{10in}
\end{frame}

\begin{frame}[label={sec:org740c6ff}]{Example}
\end{frame}

\begin{frame}[label={sec:orgbf5acec}]{Algebraic Properties of Series}
Suppose \(\sum\limits_{n=1}^{\infty} a_n\) converges to \(A\) and
\(\sum\limits_{n=1}^{\infty} b_n\) converges to \(B\), and \(c\) is a
real number.  Then we have
\begin{enumerate}
\item \(\sum\limits_{n=1}^{\infty} a_n \pm b_n =\)
\item \(\sum\limits_{n=1}^{\infty} c a_n =\)
\end{enumerate}
\vspace{10in}   
\end{frame}

\begin{frame}[label={sec:orgaa4dfb4}]{Example}
\end{frame}
\end{document}