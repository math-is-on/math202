% Created 2024-05-08 Wed 07:53
% Intended LaTeX compiler: pdflatex
\documentclass[presentation]{beamer}
\usepackage[utf8]{inputenc}
\usepackage[T1]{fontenc}
\usepackage{graphicx}
\usepackage{longtable}
\usepackage{wrapfig}
\usepackage{rotating}
\usepackage[normalem]{ulem}
\usepackage{amsmath}
\usepackage{amssymb}
\usepackage{capt-of}
\usepackage{hyperref}
\mode<beamer>{\usetheme{Madrid}}
\definecolor{SUred}{rgb}{0.59375, 0, 0.17969} % SU red (primary)
\definecolor{SUblue}{rgb}{0, 0.17578, 0.38281} % SU blue (secondary)
\setbeamercolor{palette primary}{bg=SUred,fg=white}
\setbeamercolor{palette secondary}{bg=SUblue,fg=white}
\setbeamercolor{palette tertiary}{bg=SUblue,fg=white}
\setbeamercolor{palette quaternary}{bg=SUblue,fg=white}
\setbeamercolor{structure}{fg=SUblue} % itemize, enumerate, etc
\setbeamercolor{section in toc}{fg=SUblue} % TOC sections
% Override palette coloring with secondary
\setbeamercolor{subsection in head/foot}{bg=SUblue,fg=white}
\setbeamercolor{date in head/foot}{bg=SUblue,fg=white}
\institute[SU]{Shenandoah University}
\titlegraphic{\includegraphics[width=0.5\textwidth]{\string~/Documents/suLogo/suLogo.pdf}}
\newcommand{\R}{\mathbb{R}}
\usepackage{tikz}
\usepackage{pgfplots}
\usetheme{default}
\author{Chase Mathison\thanks{cmathiso@su.edu}}
\date{8 May 2024}
\title{Even More Taylor Series!}
\hypersetup{
 pdfauthor={Chase Mathison},
 pdftitle={Even More Taylor Series!},
 pdfkeywords={},
 pdfsubject={},
 pdfcreator={Emacs 29.1 (Org mode 9.6.7)}, 
 pdflang={English}}
\begin{document}

\maketitle

\section{Announcements}
\label{sec:orga103bd9}
\begin{frame}[label={sec:org3c44db7}]{Announcements}
\begin{enumerate}
\item Final exam next Wednesday at 8am.
\item If you're behind on any assignments, make sure to get them turned
it before the final for credit.
\end{enumerate}
\end{frame}

\section{Lecture}
\label{sec:org124f4f1}
\begin{frame}[label={sec:org581a3b0}]{Using the Remainder}
Because the remainder term for a Taylor polynomial is \alert{defined} to be
\[
R_n(x) = p_n(x) - f(x)\]

We can say that the Taylor series for the function \(f\) centered at \(a\)
will converge to \(f\) (on it's interval of convergence) if
\[
\lim_{n\rightarrow \infty} R_n(x) = \hspace{1in}\]

This happens with most functions in this class.
\end{frame}

\begin{frame}[label={sec:orgb183126}]{Using the Remainder}
Find the \(n\)th Taylor polynomial centered at \(0\), \(p_{n}(x)\) for
the function \(f(x) = \cos(x)\), and the corresponding remainder term \(R_n(x).\)
Show that for any \(x\), \(\lim_{n\rightarrow \infty}R_n(x) = 0.\)
\vspace{10in}
\end{frame}

\begin{frame}[label={sec:org5d98faf}]{Working with Taylor series}
Here are some of the most important power series for you to know:
\vspace{10in}
\end{frame}

\begin{frame}[label={sec:orgae6f2cc}]{Working with Taylor series}
Find the Maclaurin series for the function \(f(x) = \cos(x^2).\) Use this to find
\(f^{(80)}(0).\)
\vspace{10in}
\end{frame}

\begin{frame}[label={sec:orgf87fce0}]{Working with Taylor series}
\end{frame}

\begin{frame}[label={sec:org9dfdd3a}]{Example}
Find the Taylor series centered at \(x = 3\) for the function \(f(x) = e^x.\)
\vspace{10in}
\end{frame}

\begin{frame}[label={sec:orgc32e411}]{Example}
\end{frame}

\begin{frame}[label={sec:org8ccabb7}]{Example}
Use a Maclaurin series to find
\[
\int\limits_0^1 e^{-x^2}\,dx\]
(Your answer will be in the form of a convergent series).
\vspace{10in}
\end{frame}

\begin{frame}[label={sec:org3397ba3}]{Example}
\end{frame}
\end{document}