% Created 2024-05-03 Fri 11:25
% Intended LaTeX compiler: pdflatex
\documentclass[presentation]{beamer}
\usepackage[utf8]{inputenc}
\usepackage[T1]{fontenc}
\usepackage{graphicx}
\usepackage{longtable}
\usepackage{wrapfig}
\usepackage{rotating}
\usepackage[normalem]{ulem}
\usepackage{amsmath}
\usepackage{amssymb}
\usepackage{capt-of}
\usepackage{hyperref}
\mode<beamer>{\usetheme{Madrid}}
\definecolor{SUred}{rgb}{0.59375, 0, 0.17969} % SU red (primary)
\definecolor{SUblue}{rgb}{0, 0.17578, 0.38281} % SU blue (secondary)
\setbeamercolor{palette primary}{bg=SUred,fg=white}
\setbeamercolor{palette secondary}{bg=SUblue,fg=white}
\setbeamercolor{palette tertiary}{bg=SUblue,fg=white}
\setbeamercolor{palette quaternary}{bg=SUblue,fg=white}
\setbeamercolor{structure}{fg=SUblue} % itemize, enumerate, etc
\setbeamercolor{section in toc}{fg=SUblue} % TOC sections
% Override palette coloring with secondary
\setbeamercolor{subsection in head/foot}{bg=SUblue,fg=white}
\setbeamercolor{date in head/foot}{bg=SUblue,fg=white}
\institute[SU]{Shenandoah University}
\titlegraphic{\includegraphics[width=0.5\textwidth]{\string~/Documents/suLogo/suLogo.pdf}}
\newcommand{\R}{\mathbb{R}}
\usetheme{default}
\author{Chase Mathison\thanks{cmathiso@su.edu}}
\date{6 May 2024}
\title{More Taylor Series!}
\hypersetup{
 pdfauthor={Chase Mathison},
 pdftitle={More Taylor Series!},
 pdfkeywords={},
 pdfsubject={},
 pdfcreator={Emacs 29.1 (Org mode 9.6.7)}, 
 pdflang={English}}
\begin{document}

\maketitle

\section{Announcements}
\label{sec:orgdad5ef1}
\begin{frame}[label={sec:org0929c71}]{Announcements}
\begin{enumerate}
\item Exam corrections, due Friday
\item New homework, due Friday
\item Final Exam, Wednesday May 15, 8:00am - 10:30am.
\item Office hours today are 12pm - 1pm.
\end{enumerate}
\end{frame}

\section{Lecture}
\label{sec:org680b3ad}
\begin{frame}[label={sec:orgd58c53e}]{A reminder of Taylor Series}
Here's what we talked about last time!  Suppose \(f(x)\) is a function that
has as many derivatives at the point \(a\) as we desire.  Then the \uline{\hspace*{1in}} for
\(f\) centered at \(a\) is given by

\vspace{1in}

If \(a = 0\), we call the series the \uline{\hspace*{1in}} for \(f\).
\end{frame}

\begin{frame}[label={sec:org391bfb0}]{Example}
Find the Maclaurin series for the function \(f(x) = e^x.\)  Also find the radius of convergence and interval of convergence.
\vspace{10in}
\end{frame}

\begin{frame}[label={sec:org3213743}]{Example}
\end{frame}

\begin{frame}[label={sec:org63f7afd}]{Example}
Find the Taylor series centered at \(a=1\) for the function \(f(x) =
\ln \left( x \right)\). Also find the radius of convergence and interval of convergence.
\vspace{10in}
\end{frame}

\begin{frame}[label={sec:org1f4785f}]{Example}
\end{frame}

\begin{frame}[label={sec:org5066af6}]{Taylor Polynomials and the Remainder}
The partial sums associated with a Taylor series is called a \uline{\hspace*{1in}}.

\begin{definition}[Taylor Polynomial]
If \(f\) has \(n\) derivatives at \(x = a\), then the \(n\)th Taylor Polynomial for \(f\) at \(a\)
is
\[
p_n(x) = \hspace{2in}\]
We also define the \(n\)th remainder as
\[
R_n(x) = \hspace{2in}\]
\end{definition}
\end{frame}

\begin{frame}[label={sec:orgd96795b}]{Example}
Find \(p_0(x),p_1(x),p_2(x),\) and \(p_3(x)\) at \(a = 0\) for the
function \(f(x) = e^x.\)  Also write down the corresponding remainders.

\vspace{10in}
\end{frame}

\begin{frame}[label={sec:org1863cf2}]{Example}
\end{frame}

\begin{frame}[label={sec:org88c8445}]{Example}
Find \(p_2(x)\) at \(a = -1\) for the function \(f(x) = \sqrt{5+x}\). What is the correponding remainder?
\vspace{10in}
\end{frame}

\begin{frame}[label={sec:org3595968}]{Example}
\end{frame}

\begin{frame}[label={sec:orgcaff2a7}]{Another way to write the remainder term}
Let's finish up by seeing another way to write the remainder term
\(R_n(x)\) for a function \(f\) that has \(n+1\) derivatives at the
point \(a\).
\vspace{10in}
\end{frame}

\begin{frame}[label={sec:org3548e9d}]{Another way to write the remainder term}
\end{frame}
\end{document}