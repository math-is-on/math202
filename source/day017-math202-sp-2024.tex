% Created 2024-02-19 Mon 11:27
% Intended LaTeX compiler: pdflatex
\documentclass[presentation]{beamer}
\usepackage[utf8]{inputenc}
\usepackage[T1]{fontenc}
\usepackage{graphicx}
\usepackage{longtable}
\usepackage{wrapfig}
\usepackage{rotating}
\usepackage[normalem]{ulem}
\usepackage{amsmath}
\usepackage{amssymb}
\usepackage{capt-of}
\usepackage{hyperref}
\mode<beamer>{\usetheme{Madrid}}
\definecolor{SUred}{rgb}{0.59375, 0, 0.17969} % SU red (primary)
\definecolor{SUblue}{rgb}{0, 0.17578, 0.38281} % SU blue (secondary)
\setbeamercolor{palette primary}{bg=SUred,fg=white}
\setbeamercolor{palette secondary}{bg=SUblue,fg=white}
\setbeamercolor{palette tertiary}{bg=SUblue,fg=white}
\setbeamercolor{palette quaternary}{bg=SUblue,fg=white}
\setbeamercolor{structure}{fg=SUblue} % itemize, enumerate, etc
\setbeamercolor{section in toc}{fg=SUblue} % TOC sections
% Override palette coloring with secondary
\setbeamercolor{subsection in head/foot}{bg=SUblue,fg=white}
\setbeamercolor{date in head/foot}{bg=SUblue,fg=white}
\institute[SU]{Shenandoah University}
\titlegraphic{\includegraphics[width=0.5\textwidth]{\string~/Documents/suLogo/suLogo.pdf}}
\usepackage{tikz}
\usetheme{default}
\author{Chase Mathison\thanks{cmathiso@su.edu}}
\date{20 February 2024}
\title{Integration by Parts, Part I}
\hypersetup{
 pdfauthor={Chase Mathison},
 pdftitle={Integration by Parts, Part I},
 pdfkeywords={},
 pdfsubject={},
 pdfcreator={Emacs 29.1 (Org mode 9.6.7)}, 
 pdflang={English}}
\begin{document}

\maketitle
\section{Announcements}
\label{sec:org3f9dd8c}
\begin{frame}[label={sec:org0c73c46}]{Announcements}
\begin{enumerate}
\item Homework in MyOpenMath.
\item Exams will be given back to you tomorrow.
\item Office hours, 10am - 11am.
\end{enumerate}
\end{frame}

\section{The lecture}
\label{sec:org3d16c63}
\begin{frame}[label={sec:org0b7e201}]{Shifting gears}
Today begins our study of new techniques of integration.  After
today's class, you be able to integrate something that looks like
\[
\int\limits_{}^{} xe^x\,dx \]
which currently we don't know how to handle.

The secret to integrating this and many other functions lies in a
technique called \uline{\hspace{1in}}.  
Integration by parts is the integration rule that corresponds to the
product rule for differentiation, so let's remind ourselves of the
product rule. 
\end{frame}

\begin{frame}[label={sec:org0e0b4d4}]{The product rule}
If \(f \left( x \right)\) and \(g \left( x \right)\) are
differentiable functions, then so is \(f \left( x \right)g \left( x
\right)\) and
\[
\hspace{1in} \]

If we integrate both sides of this, we get:
\[
\hspace{1in}\]
Rearranging this, we have one way of writing the integration by parts
formula:
\[
\hspace{1in}\]
\end{frame}

\begin{frame}[label={sec:orgccdb46f}]{The product rule}
The way this is usually written is as follows:
\[
\hspace{1in} \]
Matching corresponding pieces with the slide before, we have
\[
\hspace{1in}\]
with
\[
\hspace{1in} \]
This is not a \uline{\hspace{1in}}.

This is simply a quick way to write out integration by parts.
\end{frame}

\begin{frame}[label={sec:orgc962f23}]{Example}
Find
\[
\int\limits_{}^{} xe^x\,dx \]
\vspace{10in}
\end{frame}

\begin{frame}[label={sec:orgb87a76d}]{Example}
\end{frame}

\begin{frame}[label={sec:org097e090}]{How do I pick \(u\) and \(dv\)?}
A very natural question comes up when learning integration by parts:
Which function should we make \(u\) and which one we should make \(dv\)?

Some books teach the following acronym: LIATE.

How do we use this acronym?  When you're deciding which function to
make \(u\), you're going to choose the function that appears furthest
to the left in the acronym:
\begin{description}
\item[{L}] 

\item[{I}] 

\item[{A}] 

\item[{T}] 

\item[{E}] 
\end{description}
\end{frame}

\begin{frame}[label={sec:orgf3b371d}]{Example}
Find
\[
\int\limits_{}^{} \ln (x)\,dx \]
\vspace{10in}
\end{frame}

\begin{frame}[label={sec:orge62d4af}]{Example}
\end{frame}

\begin{frame}[label={sec:org0b4e9f8}]{Example}
Find
\[
\int\limits_{}^{} x^2 \sin \left( x \right)\,dx \]
\vspace{10in}
\end{frame}

\begin{frame}[label={sec:org15c5b61}]{Example}
\end{frame}
\end{document}