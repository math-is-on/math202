% Created 2024-03-25 Mon 12:05
% Intended LaTeX compiler: pdflatex
\documentclass[presentation]{beamer}
\usepackage[utf8]{inputenc}
\usepackage[T1]{fontenc}
\usepackage{graphicx}
\usepackage{longtable}
\usepackage{wrapfig}
\usepackage{rotating}
\usepackage[normalem]{ulem}
\usepackage{amsmath}
\usepackage{amssymb}
\usepackage{capt-of}
\usepackage{hyperref}
\mode<beamer>{\usetheme{Madrid}}
\definecolor{SUred}{rgb}{0.59375, 0, 0.17969} % SU red (primary)
\definecolor{SUblue}{rgb}{0, 0.17578, 0.38281} % SU blue (secondary)
\setbeamercolor{palette primary}{bg=SUred,fg=white}
\setbeamercolor{palette secondary}{bg=SUblue,fg=white}
\setbeamercolor{palette tertiary}{bg=SUblue,fg=white}
\setbeamercolor{palette quaternary}{bg=SUblue,fg=white}
\setbeamercolor{structure}{fg=SUblue} % itemize, enumerate, etc
\setbeamercolor{section in toc}{fg=SUblue} % TOC sections
% Override palette coloring with secondary
\setbeamercolor{subsection in head/foot}{bg=SUblue,fg=white}
\setbeamercolor{date in head/foot}{bg=SUblue,fg=white}
\institute[SU]{Shenandoah University}
\titlegraphic{\includegraphics[width=0.5\textwidth]{\string~/Documents/suLogo/suLogo.pdf}}
\newcommand{\R}{\mathbb{R}}
\usetheme{default}
\author{Chase Mathison\thanks{cmathiso@su.edu}}
\date{26 March 2024}
\title{Improper Integrals}
\hypersetup{
 pdfauthor={Chase Mathison},
 pdftitle={Improper Integrals},
 pdfkeywords={},
 pdfsubject={},
 pdfcreator={Emacs 29.1 (Org mode 9.6.7)}, 
 pdflang={English}}
\begin{document}

\maketitle

\section{Announcements}
\label{sec:orga2317fb}
\begin{frame}[label={sec:org4355fcb}]{Announcements}
\begin{enumerate}
\item Exam corrections, Due next Tuesday
\item Office hours, every day, 10am - 11am.
\item Homework in MyOpenMath.
\end{enumerate}
\end{frame}

\section{Lecture}
\label{sec:org102de56}
\begin{frame}[label={sec:org89ccb9e}]{Example}
How would we define something that looks like
\[
\int\limits_1^{\infty} \frac{1}{t^2}\,dt?\]
\vspace{10in}
\end{frame}

\begin{frame}[label={sec:org18e2d19}]{Improper integrals}
This process that we just did should remind you of a limit, because we
essentially just took a limit!

Sometimes, like in the example above, we are interested in integrating
functions not just over a finite interval \(\left[ a,b \right]\), but
over an infinite interval, like \(\left[ a,\infty \right)\).  This
gives rise to one type of \uline{\hspace{1in}}.

We'll cover the other type of improper integral tomorrow.  For now,
let's look at some definitions.
\end{frame}

\begin{frame}[label={sec:org3b42398}]{Improper integrals}
\begin{definition}[Improper Integral (infinite bound)]
If \(a\) is a real number, we define
\[
\int\limits_a^{\infty} f(x)\,dx = \hspace{2in}.\]
Similarly, we define
\[
\int\limits_{-\infty}^a f(x)\,dx = \hspace{2in}.\]
We define
\[
\int\limits_{-\infty}^{\infty} f(x)\,dx = \hspace{2in}.\]
\end{definition}
\end{frame}

\begin{frame}[label={sec:orgaf3dc2f}]{Improper Integrals}
If the above limits exist, we say the improper integral \uline{\hspace*{1in}}.
Otherwise, the improper integral \uline{\hspace*{1in}}.
\end{frame}

\begin{frame}[label={sec:org4c2cce2}]{Example}
Determine if the integral
\[
\int\limits_1^{\infty} \frac{1}{x}\,dx \]
converges.
\vspace{10in}
\end{frame}

\begin{frame}[label={sec:orgde3979a}]{Example}
\end{frame}

\begin{frame}[label={sec:org78e7807}]{Example (Gabriel's horn)}
Find the volume of revolution of the solid generated by rotating the
region whose top boundary is the graph of the function
\[y = \frac{1}{x} \]
and whose lower boundary is the \(x-\)axis for \(x \ge 1\).
\vspace{10in}
\end{frame}

\begin{frame}[label={sec:orga94118e}]{Example}
\end{frame}

\begin{frame}[label={sec:orgce354cf}]{Example}
Does the integral
\[
\int\limits_{-\infty}^0 xe^{-x^2}\,dx \]
converge or diverge?
\vspace{10in}
\end{frame}

\begin{frame}[label={sec:org7ebbf8f}]{Example}
\end{frame}

\begin{frame}[label={sec:orgafa4b45}]{Example}
Does the integral
\[
\int\limits_{-\infty}^{\infty} \frac{1}{1+x^2}\,dx \]
converge or diverge? Find its value if it converges.
\vspace{10in}
\end{frame}

\begin{frame}[label={sec:orgb5f47c1}]{Example (Important!)}
Show that
\[\int\limits_1^{\infty} \frac{1}{x^p}\,dx \]
Converges for \(p > 1\) and diverges for \(p \le 1\).
\vspace{10in}
\end{frame}

\begin{frame}[label={sec:orgfb1d57e}]{Example}
\end{frame}
\end{document}