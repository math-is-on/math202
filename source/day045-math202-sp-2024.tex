% Created 2024-04-28 Sun 22:54
% Intended LaTeX compiler: pdflatex
\documentclass[presentation]{beamer}
\usepackage[utf8]{inputenc}
\usepackage[T1]{fontenc}
\usepackage{graphicx}
\usepackage{longtable}
\usepackage{wrapfig}
\usepackage{rotating}
\usepackage[normalem]{ulem}
\usepackage{amsmath}
\usepackage{amssymb}
\usepackage{capt-of}
\usepackage{hyperref}
\mode<beamer>{\usetheme{Madrid}}
\definecolor{SUred}{rgb}{0.59375, 0, 0.17969} % SU red (primary)
\definecolor{SUblue}{rgb}{0, 0.17578, 0.38281} % SU blue (secondary)
\setbeamercolor{palette primary}{bg=SUred,fg=white}
\setbeamercolor{palette secondary}{bg=SUblue,fg=white}
\setbeamercolor{palette tertiary}{bg=SUblue,fg=white}
\setbeamercolor{palette quaternary}{bg=SUblue,fg=white}
\setbeamercolor{structure}{fg=SUblue} % itemize, enumerate, etc
\setbeamercolor{section in toc}{fg=SUblue} % TOC sections
% Override palette coloring with secondary
\setbeamercolor{subsection in head/foot}{bg=SUblue,fg=white}
\setbeamercolor{date in head/foot}{bg=SUblue,fg=white}
\institute[SU]{Shenandoah University}
\titlegraphic{\includegraphics[width=0.5\textwidth]{\string~/Documents/suLogo/suLogo.pdf}}
\newcommand{\R}{\mathbb{R}}
\usetheme{default}
\author{Chase Mathison\thanks{cmathiso@su.edu}}
\date{29 April 2024}
\title{Taylor Series!}
\hypersetup{
 pdfauthor={Chase Mathison},
 pdftitle={Taylor Series!},
 pdfkeywords={},
 pdfsubject={},
 pdfcreator={Emacs 29.1 (Org mode 9.6.7)}, 
 pdflang={English}}
\begin{document}

\maketitle

\section{Announcements}
\label{sec:org7142017}
\begin{enumerate}
\item Exam Wednesday.
\item No class Thursday.
\item Office hours 10am - 11am.
\end{enumerate}

\section{Lecture}
\label{sec:org62fe56b}
\begin{frame}[label={sec:orgac33696}]{Taylor Series}
So we know how to find a power series for a function that's related to
\(\frac{a}{1-r},\) but how would we find a power series for the
function \(f(x) = \sin(x)\) centered at \(0\)?  \vspace{10in}
\end{frame}

\begin{frame}[label={sec:orge47b20c}]{Taylor Series}
\end{frame}

\begin{frame}[label={sec:org271d911}]{Taylor Series}
\end{frame}

\begin{frame}[label={sec:org8790c72}]{Taylor Series}
Given a function \(f(x)\) with as many derivatives as we want at \(x = a\), the \uline{\hspace*{1in}} for
\(f\) centered at \(a\) is defined to be the following series:

\vspace{1in}

If \(a = 0\), we call this series the \uline{\hspace*{1in}} for \(f\) instead.
\end{frame}

\begin{frame}[label={sec:orga52c054}]{Example}
Use the Maclaurin series of \(\sin(x)\) that we just found to find the
Maclaurin series for \(\cos(x).\)
\vspace{10in}
\end{frame}
\end{document}