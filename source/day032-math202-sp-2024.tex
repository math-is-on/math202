% Created 2024-03-28 Thu 14:15
% Intended LaTeX compiler: pdflatex
\documentclass[presentation]{beamer}
\usepackage[utf8]{inputenc}
\usepackage[T1]{fontenc}
\usepackage{graphicx}
\usepackage{longtable}
\usepackage{wrapfig}
\usepackage{rotating}
\usepackage[normalem]{ulem}
\usepackage{amsmath}
\usepackage{amssymb}
\usepackage{capt-of}
\usepackage{hyperref}
\mode<beamer>{\usetheme{Madrid}}
\definecolor{SUred}{rgb}{0.59375, 0, 0.17969} % SU red (primary)
\definecolor{SUblue}{rgb}{0, 0.17578, 0.38281} % SU blue (secondary)
\setbeamercolor{palette primary}{bg=SUred,fg=white}
\setbeamercolor{palette secondary}{bg=SUblue,fg=white}
\setbeamercolor{palette tertiary}{bg=SUblue,fg=white}
\setbeamercolor{palette quaternary}{bg=SUblue,fg=white}
\setbeamercolor{structure}{fg=SUblue} % itemize, enumerate, etc
\setbeamercolor{section in toc}{fg=SUblue} % TOC sections
% Override palette coloring with secondary
\setbeamercolor{subsection in head/foot}{bg=SUblue,fg=white}
\setbeamercolor{date in head/foot}{bg=SUblue,fg=white}
\institute[SU]{Shenandoah University}
\titlegraphic{\includegraphics[width=0.5\textwidth]{\string~/Documents/suLogo/suLogo.pdf}}
\newcommand{\R}{\mathbb{R}}
\usetheme{default}
\author{Chase Mathison\thanks{cmathiso@su.edu}}
\date{1 April 2024}
\title{Sequences}
\hypersetup{
 pdfauthor={Chase Mathison},
 pdftitle={Sequences},
 pdfkeywords={},
 pdfsubject={},
 pdfcreator={Emacs 29.1 (Org mode 9.6.7)}, 
 pdflang={English}}
\begin{document}

\maketitle

\section{Announcements}
\label{sec:orgf89c34f}
\begin{frame}[label={sec:org5ee81af}]{Announcements}
\begin{enumerate}
\item Homework due tonight.
\item Homework/Project in Canvas.
\item Exam corrections due Tuesday.
\end{enumerate}
\end{frame}

\section{Lecture}
\label{sec:orge8fc7bd}
\begin{frame}[label={sec:org948a8cc}]{Sequences}
\begin{definition}[Infinite Sequence]
A \uline{\hspace*{1in}} \(\left\{ a_n \right\}\)is an \uline{\hspace*{1in}} of numbers of
the form
\[
\hspace{1in}\]
The subscript \(n\) is called the \uline{\hspace*{1in}} and usually (but not always) begins
at \(n = \hspace{1in}\). Each number \(a_n\) is called a \uline{\hspace*{1in}} in the sequence.
\end{definition}
\end{frame}

\begin{frame}[label={sec:orgac073c5}]{Example}
Write out the first 4 terms in the following examples of sequences:
\begin{enumerate}
\item \(a_n = \frac{1}{n}\) with \(n \ge 1\) (Explicit formula).
\item \(b_{n+2} = b_{n+1} + b_{n}\) with \(n \ge 0\) and \(b_0 = 1\) and \(b_1 = 1\). (Recurrence Relation)
\item \(c_n = \frac{1}{2^n}\), \(n \ge 3\)
\item \(d_n = d_{n-1} + 5\), with \(n \ge 1\), \(d_1 = 2\) (Arithmetic)
\item \(e_{n+1} = \frac{1}{3}e_n\), with \(n \ge 0\), \(e_0 = 2\) (Geometric)
\end{enumerate}
\vspace{10in}
\end{frame}

\begin{frame}[label={sec:org3cc7a04}]{Example}
\end{frame}

\begin{frame}[label={sec:org067ddd2}]{Example}
Find an explicit formula for the sequences
\begin{enumerate}
\item \[\frac{1}{3}, \frac{1}{2}, \frac{3}{5}, \frac{2}{3}, \frac{5}{7}, \frac{3}{4},\ldots\]
\item \[4,5,7,11,19,35,\ldots\]
\end{enumerate}
\vspace{10in}   
\end{frame}

\begin{frame}[label={sec:org70bf242}]{The Limit of a Sequence}
We'll be interested in what happens to a sequence as \(n \rightarrow
\infty.\) If a sequence ``settles down'' to 1 number as \(n \rightarrow
\infty\), then we'll call that number the \uline{\hspace*{1in}} of the sequence.

\begin{definition}[Limit of a sequence]
If \(\left\{ a_n \right\}_{n=n_0}^{\infty}\) is a sequence of real
numbers, we call \(L\) the \alert{limit} of the sequence \(\left\{ a_n
\right\}_{n=n_0}^{\infty}\) if, given \(\epsilon > 0\) there is some integer \(N\) such that if \(n \ge N\), then
\[
\left| a_n - L \right| < \epsilon.\]

If the sequence \(\left\{ a_n \right\}_{n=n_0}^{\infty}\) has a limit \(L\), then we write
\[
\lim_{n\rightarrow\infty} a_n = L\]
and also say that the sequence \alert{converges} to the limit \(L\).
\end{definition}
\end{frame}


\begin{frame}[label={sec:org1d125e1}]{Example}
Show, using the definition of the limit of a sequence, that
\[
\lim_{n\rightarrow\infty} \frac{1}{2^n} = 0.\]

\vspace{10in}
\end{frame}

\begin{frame}[label={sec:orgccd95f6}]{Example}
\end{frame}

\begin{frame}[label={sec:org626af4b}]{Example}
Show that the sequence \(b_n = \frac{n-1}{n}\) for \(n \ge 1\)
converges to \(1\) using the definition of the limit of a sequence.

\vspace{10in}
\end{frame}
\end{document}