% Created 2024-04-14 Sun 22:41
% Intended LaTeX compiler: pdflatex
\documentclass[presentation]{beamer}
\usepackage[utf8]{inputenc}
\usepackage[T1]{fontenc}
\usepackage{graphicx}
\usepackage{longtable}
\usepackage{wrapfig}
\usepackage{rotating}
\usepackage[normalem]{ulem}
\usepackage{amsmath}
\usepackage{amssymb}
\usepackage{capt-of}
\usepackage{hyperref}
\mode<beamer>{\usetheme{Madrid}}
\definecolor{SUred}{rgb}{0.59375, 0, 0.17969} % SU red (primary)
\definecolor{SUblue}{rgb}{0, 0.17578, 0.38281} % SU blue (secondary)
\setbeamercolor{palette primary}{bg=SUred,fg=white}
\setbeamercolor{palette secondary}{bg=SUblue,fg=white}
\setbeamercolor{palette tertiary}{bg=SUblue,fg=white}
\setbeamercolor{palette quaternary}{bg=SUblue,fg=white}
\setbeamercolor{structure}{fg=SUblue} % itemize, enumerate, etc
\setbeamercolor{section in toc}{fg=SUblue} % TOC sections
% Override palette coloring with secondary
\setbeamercolor{subsection in head/foot}{bg=SUblue,fg=white}
\setbeamercolor{date in head/foot}{bg=SUblue,fg=white}
\institute[SU]{Shenandoah University}
\titlegraphic{\includegraphics[width=0.5\textwidth]{\string~/Documents/suLogo/suLogo.pdf}}
\newcommand{\R}{\mathbb{R}}
\usetheme{default}
\author{Chase Mathison\thanks{cmathiso@su.edu}}
\date{11 April 2024}
\title{The Comparison Tests}
\hypersetup{
 pdfauthor={Chase Mathison},
 pdftitle={The Comparison Tests},
 pdfkeywords={},
 pdfsubject={},
 pdfcreator={Emacs 29.1 (Org mode 9.6.7)}, 
 pdflang={English}}
\begin{document}

\maketitle

\section{Announcements}
\label{sec:orge345e4c}
\begin{frame}[label={sec:org453fc95}]{Announcements}
\begin{enumerate}
\item Homework due, and new homework in MyOpenMath.
\item Office hours, 10am - 11am.
\end{enumerate}
\end{frame}

\section{Lecture}
\label{sec:orge41902b}
\begin{frame}[label={sec:org58cf061}]{The Comparison Test}
Now that we have some series for which we can describe their
convergence/divergence, let's look at another test for convergence!
\end{frame}

\begin{frame}[label={sec:orgac27d25}]{The Comparison Test}
Let \(\sum\limits_{n=1}^{\infty} a_n\) be a series with \(a_n \ge 0\) for all \(n\), and suppose
\(\sum\limits_{n=1}^{\infty} b_n\) is another series which we know is convergent.

Then, if there is an \(N\) such that \(a_n\) \uline{\hspace*{0.25in}} \(b_n\) for all \(n \ge N\), then
\[
\sum\limits_{n=1}^{\infty} a_n\]
\uline{\hspace*{1in}}.

Similarly, if \(\sum\limits_{n=1}^{\infty} b_n\) is a series which we know is divergent,
and there is an \(N\) such that \(a_n\) \uline{\hspace*{0.25in}} \(b_n\) for all \(n \ge N\), then
\[
\sum\limits_{n=1}^{\infty} a_n\]
\uline{\hspace*{1in}}.
\end{frame}

\begin{frame}[label={sec:org1530b07}]{Example}
Discuss the convergence/divergence of
\[
\sum\limits_{n=1}^{\infty} \frac{1}{n!}.\]
\vspace{10in}
\end{frame}

\begin{frame}[label={sec:org9f4624e}]{Example}
Discuss the convergence/divergence of
\[
\sum\limits_{n=1}^{\infty} \frac{\sin^2(n)}{n^2}\]
\vspace{10in}
\end{frame}

\begin{frame}[label={sec:orgf86e12e}]{Example}
Discuss the convergence/divergence of
\[
\sum\limits_{n=2}^{\infty} \frac{1}{\ln(n)}\]
\vspace{10in}
\end{frame}

\begin{frame}[label={sec:org441e1b5}]{The Limit Comparison Test}
The comparison test is nice, but it's a little too simplistic sometimes.  For instance,
if we want to examine the convergence/divergence of the series
\[
\sum\limits_{n=2}^{\infty} \frac{1}{n^2-1}\]
then our natural instinct is to compare this series to \(\sum\limits_{n=2}^{\infty} \frac{1}{n^2},\)
but
\[
\frac{1}{n^2-1} \hspace{1in} \frac{1}{n^2}\]
so the comparison test wouldn't tell us anything here.  This is where the limit comparison
test is useful!
\end{frame}

\begin{frame}[label={sec:org70f04c3}]{The Limit Comparison Test}
Suppose \(\sum\limits_{n=1}^{\infty} a_n\) and \(\sum\limits_{n=1}^{\infty} b_n\) are series with \(a_n,b_n \ge 0\).
Let \(M = \lim_{n\rightarrow \infty} \frac{a_n}{b_n}\).  Then
\begin{enumerate}
\item If \(M > 0\), is a real number, then
\(\sum\limits_{n=1}^{\infty} a_n \text{ and } \sum\limits_{n=1}^{\infty} b_n\)
have the \uline{\hspace*{1in}}.
\item If \(M = 0\) and \(\sum\limits_{n=1}^{\infty} b_n\) converges, then
\(\sum\limits_{n=1}^{\infty} a_n\)
\uline{\hspace*{1in}}.
\item If \(M = \infty\) and \(\sum\limits_{n=1}^{\infty} b_n\) diverges, then
\(\sum\limits_{n=1}^{\infty} a_n\)
\uline{\hspace*{1in}}.
\end{enumerate}
\end{frame}

\begin{frame}[label={sec:org610709b}]{Example}
Use the limit comparison test to discuss the convergence/divergence of the series
\[
\sum\limits_{n=2}^{\infty} \frac{1}{n^2-1}\]
\vspace{10in}
\end{frame}

\begin{frame}[label={sec:orgce3b004}]{Example}
Use the limit comparison test to discuss the convergence/divergence of the series
\[
\sum\limits_{n=2}^{\infty} \frac{1}{2^{\ln(\ln(n))}}\]
\vspace{10in}
\end{frame}

\begin{frame}[label={sec:orge96cc1b}]{Example}
\end{frame}
\end{document}