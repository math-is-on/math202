% Created 2024-01-31 Wed 10:41
% Intended LaTeX compiler: pdflatex
\documentclass[presentation]{beamer}
\usepackage[utf8]{inputenc}
\usepackage[T1]{fontenc}
\usepackage{graphicx}
\usepackage{longtable}
\usepackage{wrapfig}
\usepackage{rotating}
\usepackage[normalem]{ulem}
\usepackage{amsmath}
\usepackage{amssymb}
\usepackage{capt-of}
\usepackage{hyperref}
\mode<beamer>{\usetheme{Madrid}}
\definecolor{SUred}{rgb}{0.59375, 0, 0.17969} % SU red (primary)
\definecolor{SUblue}{rgb}{0, 0.17578, 0.38281} % SU blue (secondary)
\setbeamercolor{palette primary}{bg=SUred,fg=white}
\setbeamercolor{palette secondary}{bg=SUblue,fg=white}
\setbeamercolor{palette tertiary}{bg=SUblue,fg=white}
\setbeamercolor{palette quaternary}{bg=SUblue,fg=white}
\setbeamercolor{structure}{fg=SUblue} % itemize, enumerate, etc
\setbeamercolor{section in toc}{fg=SUblue} % TOC sections
% Override palette coloring with secondary
\setbeamercolor{subsection in head/foot}{bg=SUblue,fg=white}
\setbeamercolor{date in head/foot}{bg=SUblue,fg=white}
\institute[SU]{Shenandoah University}
\titlegraphic{\includegraphics[width=0.5\textwidth]{\string~/Documents/suLogo/suLogo.pdf}}
\newcommand{\R}{\mathbb{R}}
\usepackage{tikz}
\usepackage{pgfplots}
\usetheme{default}
\author{Chase Mathison\thanks{cmathiso@su.edu}}
\date{1 February 2024}
\title{Volumes by Slicing}
\hypersetup{
 pdfauthor={Chase Mathison},
 pdftitle={Volumes by Slicing},
 pdfkeywords={},
 pdfsubject={},
 pdfcreator={Emacs 29.1 (Org mode 9.6.7)}, 
 pdflang={English}}
\begin{document}

\maketitle

\section{Announcements}
\label{sec:org7bf6cec}
\begin{frame}[label={sec:orga16ce3e}]{Announcements}
\begin{enumerate}
\item Homework in MyOpenMath.
\item Quiz in Canvas.
\item Office hours: 10am - 11am.
\end{enumerate}
\end{frame}

\section{Volumes by slicing}
\label{sec:org039c35b}
\begin{frame}[label={sec:org0639fcf}]{Volumes by slicing}
Another application of integration is in finding volumes if we know
the cross sectional area of an object.  As an example, let's look at a
right circular cone with radius \(r = 3\) units, and height \(h =
10\) units.

If we take cross sections of this cylinder that are parallel to the
circular base, we get a constant cross section (which is what???).

Let's use the ``little slice'' method to figure out the volume of this
shape!
\end{frame}

\begin{frame}[label={sec:orgcc50f50}]{Volumes by slicing}
\end{frame}

\begin{frame}[label={sec:org5a8c14c}]{Volumes by slicing}
We can apply this same idea even when the cross sectional area is not
constant, but is known.  Let's use this idea to find the volume of a
pyramid with a square base (of side length 4 units) and a height of 5
units.
\vspace{10in}
\end{frame}

\begin{frame}[label={sec:org34f88bf}]{Volumes by slicing}
\end{frame}

\begin{frame}[label={sec:org784bfe2}]{Volumes by slicing: Problem solving strategy}
We've just illustrated the ``little slice method'' for volumes. To apply it in general,
use the following strategy:
\begin{enumerate}
\item Examine the solid of interest and determine the shape of a
\uline{\hspace*{1in}} of the object with respect to a certain
direction. (You should always draw pictures at this step!)
\item Determine a formula for the cross-sectional \uline{\hspace*{1in}}.
\item \uline{\hspace*{1in}} the area over the appropriate interval to find the
volume.
\end{enumerate}
\end{frame}

\begin{frame}[label={sec:orgd9a4d23}]{Example}
You want to design a building with a floor that is an isosceles
triangle with base 6 units and legs 5 units.  The top of the building
will be parabolas that connect the sides of equal length of the
triangle, with heights that are equal to their widths.  What is the
total volume of this building?
\vspace{10in}
\end{frame}

\begin{frame}[label={sec:org4000f9e}]{Example}
\end{frame}

\begin{frame}[label={sec:org58cc827}]{Example}
\end{frame}
\end{document}