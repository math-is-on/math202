% Created 2024-03-26 Tue 10:50
% Intended LaTeX compiler: pdflatex
\documentclass[presentation]{beamer}
\usepackage[utf8]{inputenc}
\usepackage[T1]{fontenc}
\usepackage{graphicx}
\usepackage{longtable}
\usepackage{wrapfig}
\usepackage{rotating}
\usepackage[normalem]{ulem}
\usepackage{amsmath}
\usepackage{amssymb}
\usepackage{capt-of}
\usepackage{hyperref}
\mode<beamer>{\usetheme{Madrid}}
\definecolor{SUred}{rgb}{0.59375, 0, 0.17969} % SU red (primary)
\definecolor{SUblue}{rgb}{0, 0.17578, 0.38281} % SU blue (secondary)
\setbeamercolor{palette primary}{bg=SUred,fg=white}
\setbeamercolor{palette secondary}{bg=SUblue,fg=white}
\setbeamercolor{palette tertiary}{bg=SUblue,fg=white}
\setbeamercolor{palette quaternary}{bg=SUblue,fg=white}
\setbeamercolor{structure}{fg=SUblue} % itemize, enumerate, etc
\setbeamercolor{section in toc}{fg=SUblue} % TOC sections
% Override palette coloring with secondary
\setbeamercolor{subsection in head/foot}{bg=SUblue,fg=white}
\setbeamercolor{date in head/foot}{bg=SUblue,fg=white}
\institute[SU]{Shenandoah University}
\titlegraphic{\includegraphics[width=0.5\textwidth]{\string~/Documents/suLogo/suLogo.pdf}}
\newcommand{\R}{\mathbb{R}}
\usetheme{default}
\author{Chase Mathison\thanks{cmathiso@su.edu}}
\date{21 March 2024}
\title{The Other Improper Integral}
\hypersetup{
 pdfauthor={Chase Mathison},
 pdftitle={The Other Improper Integral},
 pdfkeywords={},
 pdfsubject={},
 pdfcreator={Emacs 29.1 (Org mode 9.6.7)}, 
 pdflang={English}}
\begin{document}

\maketitle

\section{Announcements}
\label{sec:org220291f}
\begin{frame}[label={sec:org156a6bc}]{Announcements}
\begin{enumerate}
\item Homework
\item Corrections
\item Office Hours
\end{enumerate}
\end{frame}

\section{Lecture}
\label{sec:orgbc75d85}
\begin{frame}[label={sec:org564bd1c}]{The other improper integral}
Now let's examine a seemingly simple integral:
\[
\int\limits_{-1}^1 \frac{1}{x^2}\,dx\]
\vspace{10in}
\end{frame}

\begin{frame}[label={sec:org6c7ee70}]{The other improper integral}
\end{frame}

\begin{frame}[label={sec:org3f37e5c}]{The other improper integral}
Clearly the issue is that we're trying to integrate a function that
has a \uline{\hspace*{1in}} at \(x = 0\).  This gives rise to the second
type of improper integral.
\end{frame}

\begin{frame}[label={sec:org547566b}]{The other improper integral}
\begin{enumerate}
\item Suppose \(f(x)\) is continuous on the interval \(\left[ a,b
   \right)\), then \[ \int\limits_a^b f(x)\,dx = \hspace{2in}\]

\item Suppose \(f(x)\) is continuous on the interval \(\left( a,b
   \right]\), then \[ \int\limits_a^b f(x)\,dx= \hspace{2in}\]

\item Suppose \(f(x)\) is continuous on the interval \(\left[ a,b
   \right]\), except at the point \(x = c\), then \[ \int\limits_a^b f(x)\,dx =
   \hspace{2in}\]
\end{enumerate}
\end{frame}

\begin{frame}[label={sec:org8da119b}]{The other improper integral}
Just like before, if the limits in the previous slide exist, we say
the improper integral \uline{\hspace*{1in}} with the same value as the
limit.  If the limit fails to exists, we say the improper integral
\uline{\hspace*{1in}}.
\end{frame}

\begin{frame}[label={sec:org2bfd86c}]{Example}
Does the improper integral
\[
\int\limits_0^1 \frac{1}{\sqrt{x}}dx
\]
converge or diverge?
\vspace{10in}
\end{frame}

\begin{frame}[label={sec:org8494bd2}]{Example}
\end{frame}

\begin{frame}[label={sec:org928cb35}]{Example}
Let's take a look at the example
\[
\int\limits_{-1}^1 \frac{1}{x^2}\,dx
\]
one more time.
\vspace{10in}
\end{frame}

\begin{frame}[label={sec:orgf347576}]{Example}
\end{frame}

\begin{frame}[label={sec:org31ffc86}]{Example}
What is the value of \(a\) that gives

\[
\int\limits_0^1 \frac{1}{x^a}\,dx = 2.5?\]
\vspace{10in}
\end{frame}

\begin{frame}[label={sec:org777bece}]{Example}
\end{frame}
\end{document}