% Created 2024-03-27 Wed 11:42
% Intended LaTeX compiler: pdflatex
\documentclass[presentation]{beamer}
\usepackage[utf8]{inputenc}
\usepackage[T1]{fontenc}
\usepackage{graphicx}
\usepackage{longtable}
\usepackage{wrapfig}
\usepackage{rotating}
\usepackage[normalem]{ulem}
\usepackage{amsmath}
\usepackage{amssymb}
\usepackage{capt-of}
\usepackage{hyperref}
\mode<beamer>{\usetheme{Madrid}}
\definecolor{SUred}{rgb}{0.59375, 0, 0.17969} % SU red (primary)
\definecolor{SUblue}{rgb}{0, 0.17578, 0.38281} % SU blue (secondary)
\setbeamercolor{palette primary}{bg=SUred,fg=white}
\setbeamercolor{palette secondary}{bg=SUblue,fg=white}
\setbeamercolor{palette tertiary}{bg=SUblue,fg=white}
\setbeamercolor{palette quaternary}{bg=SUblue,fg=white}
\setbeamercolor{structure}{fg=SUblue} % itemize, enumerate, etc
\setbeamercolor{section in toc}{fg=SUblue} % TOC sections
% Override palette coloring with secondary
\setbeamercolor{subsection in head/foot}{bg=SUblue,fg=white}
\setbeamercolor{date in head/foot}{bg=SUblue,fg=white}
\institute[SU]{Shenandoah University}
\titlegraphic{\includegraphics[width=0.5\textwidth]{\string~/Documents/suLogo/suLogo.pdf}}
\newcommand{\R}{\mathbb{R}}
\usetheme{default}
\author{Chase Mathison\thanks{cmathiso@su.edu}}
\date{28 March 2024}
\title{Improper Integrals, One Last Time}
\hypersetup{
 pdfauthor={Chase Mathison},
 pdftitle={Improper Integrals, One Last Time},
 pdfkeywords={},
 pdfsubject={},
 pdfcreator={Emacs 29.1 (Org mode 9.6.7)}, 
 pdflang={English}}
\begin{document}

\maketitle

\section{Announcements}
\label{sec:orgac7468f}
\begin{frame}[label={sec:orgeb520a0}]{Announcements}
\begin{enumerate}
\item Homework in MyOpenMath.
\item Exam corrections
\end{enumerate}
\end{frame}

\section{Lecture}
\label{sec:org2a84480}
\begin{frame}[label={sec:orgcc72042}]{A Comparison Theorem}
Let's say I want to know if
\[
\int\limits_1^{\infty} \frac{1}{x^2 + x}\,dx\]
converges or diverges.  Can we say anything about convergence without
actually evaluating this integral?

\vspace{10in}
\end{frame}

\begin{frame}[label={sec:org211a2a4}]{A Comparison Theorem}
Suppose that \(0 \le f(x) \le g(x)\) for all \(x \in \left( a,\infty \right)\). Then
\begin{itemize}
\item If
\(\int\limits_a^{\infty} f(x)\,dx\)
\uline{\hspace*{1in}}, we can say that
\[
  \int\limits_a^{\infty} g(x)\,dx \hspace{2in}\]
\item If
\(\int\limits_a^{\infty} g(x)\,dx\)
\uline{\hspace*{1in}}, we can say that
\[
  \int\limits_a^{\infty} f(x)\,dx \hspace{2in}\]
\end{itemize}
\end{frame}

\begin{frame}[label={sec:org8b8b07f}]{Example}
Does the following integral converge or diverge?

\[
\int\limits_1^{\infty} \frac{1}{x^3 + x^2 + x + 1}\,dx\]
\vspace{10in}
\end{frame}

\begin{frame}[label={sec:org2f60137}]{Example}
Does the following integral converge or diverge?

\[
\int\limits_1^{\infty} \frac{1+t}{1+t^2}\,dt\]
\vspace{10in}
\end{frame}
\end{document}