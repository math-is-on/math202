% Created 2024-04-01 Mon 12:10
% Intended LaTeX compiler: pdflatex
\documentclass[presentation]{beamer}
\usepackage[utf8]{inputenc}
\usepackage[T1]{fontenc}
\usepackage{graphicx}
\usepackage{longtable}
\usepackage{wrapfig}
\usepackage{rotating}
\usepackage[normalem]{ulem}
\usepackage{amsmath}
\usepackage{amssymb}
\usepackage{capt-of}
\usepackage{hyperref}
\mode<beamer>{\usetheme{Madrid}}
\definecolor{SUred}{rgb}{0.59375, 0, 0.17969} % SU red (primary)
\definecolor{SUblue}{rgb}{0, 0.17578, 0.38281} % SU blue (secondary)
\setbeamercolor{palette primary}{bg=SUred,fg=white}
\setbeamercolor{palette secondary}{bg=SUblue,fg=white}
\setbeamercolor{palette tertiary}{bg=SUblue,fg=white}
\setbeamercolor{palette quaternary}{bg=SUblue,fg=white}
\setbeamercolor{structure}{fg=SUblue} % itemize, enumerate, etc
\setbeamercolor{section in toc}{fg=SUblue} % TOC sections
% Override palette coloring with secondary
\setbeamercolor{subsection in head/foot}{bg=SUblue,fg=white}
\setbeamercolor{date in head/foot}{bg=SUblue,fg=white}
\institute[SU]{Shenandoah University}
\titlegraphic{\includegraphics[width=0.5\textwidth]{\string~/Documents/suLogo/suLogo.pdf}}
\newcommand{\R}{\mathbb{R}}
\usetheme{default}
\author{Chase Mathison\thanks{cmathiso@su.edu}}
\date{2 April 2024}
\title{Limits of Sequences}
\hypersetup{
 pdfauthor={Chase Mathison},
 pdftitle={Limits of Sequences},
 pdfkeywords={},
 pdfsubject={},
 pdfcreator={Emacs 29.1 (Org mode 9.6.7)}, 
 pdflang={English}}
\begin{document}

\maketitle

\section{Announcements}
\label{sec:org1e4e1d3}
\begin{frame}[label={sec:org71ffac0}]{Announcements}
\begin{enumerate}
\item Homework!
\item Exam Corrections!
\item Project!
\end{enumerate}
\end{frame}

\section{Lecture}
\label{sec:org81e4bde}
\begin{frame}[label={sec:org50efd67}]{The Limit of a Sequence}
With sequences, we are usually interested in what happens in what's
known as the \uline{\hspace*{1in}} of the sequence (i.e. end behaviour):

\begin{definition}[Limit of a Sequence]
Suppose \(\left\{ a_n \right\}\) is a sequence of real numbers.  When
we say
\[
\lim_{n\rightarrow \infty} a_n = L\]
we mean that we can make \(a_n\) as close to \(L\) as we like by taking
\(n\) to be ``large enough''.  If such an \(L\) exists, we call the sequence
\(a_n\) \uline{\hspace*{1in}}.  If no such \(L\) exists, we call the sequence \uline{\hspace*{1in}}.
\end{definition}

Let's make some of these ideas a little more precise.
\end{frame}

\begin{frame}[label={sec:orgf2a1f97}]{The Limit of a Sequence}
\end{frame}

\begin{frame}[label={sec:org835f1bd}]{Limit Law}
If \(a_n = f(n)\) for some function \(f\) for all \(n \ge 1\) (or some
starting index) then if there exists \(L\) such that
\(\lim_{x\rightarrow \infty} f(x) = L\), then it must be the case that
\(\lim_{n\rightarrow \infty} a_n =\) \uline{\hspace*{1in}}.

\vspace{1in}
\end{frame}

\begin{frame}[label={sec:orgfb8a629}]{Example}
Evaluate the limits of the sequences:
\begin{enumerate}
\item \(a_n = \frac{1}{2^n}\)
\item \(b_n = \left( -1 \right)^n\)
\end{enumerate}
\vspace{10in}
\end{frame}

\begin{frame}[label={sec:org9ec20de}]{Limit Laws}
Let \(\left\{ a_n \right\}\) and \(\left\{ b_n \right\}\) be sequences such that
\(\lim_{n\rightarrow \infty} a_n  = A\) and \(\lim_{n\rightarrow\infty} b_n = B,\)  where
\(A\) and \(B\) are real numbers.  Let \(c\) be a real number.  Then the following
limit laws hold:
\begin{enumerate}
\item \(\lim_{n\rightarrow \infty} c =\)
\item \(\lim_{n\rightarrow \infty}ca_n =\)
\item \(\lim_{n\rightarrow \infty} \left( a_n \pm b_n \right) =\)
\item \(\lim_{n\rightarrow \infty} (a_nb_n) =\)
\item \(\lim_{n\rightarrow \infty} \left( \frac{a_n}{b_n} \right) =\)
\end{enumerate}

\vspace{10in}
\end{frame}

\begin{frame}[label={sec:orgff3beef}]{Example}
Evaluate
\[\lim_{k\rightarrow \infty} \frac{1 - r^k}{1 - r} \]
(Your answer will depend on the value of \(r\).)
\vspace{10in}
\end{frame}

\begin{frame}[label={sec:org7b2678d}]{Example}
Evaluate
\[
   \lim_{m\rightarrow \infty} \left( 1 - \frac{2}{m} \right)^m.\]
\vspace{10in}
\end{frame}

\begin{frame}[label={sec:org98ebefb}]{Example}
\end{frame}

\begin{frame}[label={sec:orga405e5b}]{2 More Important Theorems}
We'll take the following theorems without proof:
\begin{theorem}[Continuous Functions and Convergent Sequences]
Suppose \(\left\{ a_n \right\}\) is a convergent sequence that
converges to \(L\) and \(f\) is a function of a real variable that is
continuous at \(L\).  Then, the sequence \(\left\{ f \left( a_n \right) \right\}\)
is \uline{\hspace*{1in}} with limit \uline{\hspace*{1in}}.
\end{theorem}
\end{frame}

\begin{frame}[label={sec:org42f3b7e}]{2 More Important Theorems}
\begin{theorem}[Squeeze Theorem]
Suppose \(\left\{ a_n \right\},\) \(\left\{ b_n \right\}\) and \(\left\{ c_n \right\}\)
are all sequences that satisfy
\[
a_n \le b_n \le c_n\]
for all \(n \ge 1\) (or for all \(n\) greater than some initial index).  If
\[
\lim_{n\rightarrow \infty} a_n = L\]
and
\[
\lim_{n\rightarrow \infty} c_n = L\]
Then
\[
\lim_{n\rightarrow \infty} b_n = \hspace{1in}\]
\end{theorem}
\end{frame}

\begin{frame}[label={sec:orgaef0936}]{Example}
Use the squeeze theorem to show
\[
\lim_{k\rightarrow \infty} \frac{\sin k}{k} = 0
\]
\vspace{10in}
\end{frame}

\begin{frame}[label={sec:orgfd2b24b}]{Example}
Let
\[
S_k = 1 + \frac{1}{2} + \ldots + \frac{1}{2^k}\]
Let's try to find
\begin{enumerate}
\item A ``nicer'' way to write \(S_k\) and
\item \(\lim_{k\rightarrow \infty} S_k\)
\end{enumerate}
\vspace{10in}
\end{frame}
\end{document}