% Created 2024-04-16 Tue 07:55
% Intended LaTeX compiler: pdflatex
\documentclass[presentation]{beamer}
\usepackage[utf8]{inputenc}
\usepackage[T1]{fontenc}
\usepackage{graphicx}
\usepackage{longtable}
\usepackage{wrapfig}
\usepackage{rotating}
\usepackage[normalem]{ulem}
\usepackage{amsmath}
\usepackage{amssymb}
\usepackage{capt-of}
\usepackage{hyperref}
\mode<beamer>{\usetheme{Madrid}}
\definecolor{SUred}{rgb}{0.59375, 0, 0.17969} % SU red (primary)
\definecolor{SUblue}{rgb}{0, 0.17578, 0.38281} % SU blue (secondary)
\setbeamercolor{palette primary}{bg=SUred,fg=white}
\setbeamercolor{palette secondary}{bg=SUblue,fg=white}
\setbeamercolor{palette tertiary}{bg=SUblue,fg=white}
\setbeamercolor{palette quaternary}{bg=SUblue,fg=white}
\setbeamercolor{structure}{fg=SUblue} % itemize, enumerate, etc
\setbeamercolor{section in toc}{fg=SUblue} % TOC sections
% Override palette coloring with secondary
\setbeamercolor{subsection in head/foot}{bg=SUblue,fg=white}
\setbeamercolor{date in head/foot}{bg=SUblue,fg=white}
\institute[SU]{Shenandoah University}
\titlegraphic{\includegraphics[width=0.5\textwidth]{\string~/Documents/suLogo/suLogo.pdf}}
\newcommand{\R}{\mathbb{R}}
\usepackage{tikz}
\usepackage{pgfplots}
\usetheme{default}
\author{Chase Mathison\thanks{cmathiso@su.edu}}
\date{16 April 2024}
\title{Alternating Series}
\hypersetup{
 pdfauthor={Chase Mathison},
 pdftitle={Alternating Series},
 pdfkeywords={},
 pdfsubject={},
 pdfcreator={Emacs 29.1 (Org mode 9.6.7)}, 
 pdflang={English}}
\begin{document}

\maketitle

\section{Announcements}
\label{sec:orgf1c827c}
\begin{frame}[label={sec:org1eadf8d}]{Announcements}
\begin{enumerate}
\item Homework in M.O.M.
\item Office hours today 10am - 11am.
\item New project in Canvas.
\end{enumerate}
\end{frame}

\section{Lecture}
\label{sec:org3dd32c7}
\begin{frame}[label={sec:orga7f09e0}]{Alternating Series}
So far, the tests we've used only work on series with \uline{\hspace*{1in}} terms.

But in real life, there are series that have both positive and negative terms.

A special type of sequence of this sort is called an \uline{\hspace*{1in}}.

\begin{definition}[Alternating Series]
Any series whose terms alternate between positive and negative values is called
an alternating series.  An alternating series can be written in the forms:
\vspace{1in}
\end{definition}
\end{frame}

\begin{frame}[label={sec:org9fbb0c7}]{Example}
Which of the following are alternating series?
\begin{enumerate}
\item \(\sum\limits_{n=1}^{\infty} \left( -1 \right)^n\)
\item \(\sum\limits_{n=1}^{\infty} \left( \frac{2}{3} \right)^{n-1}\)
\item \(\sum\limits_{n=1}^{\infty} \frac{\sin \left( \frac{(2n+1)\pi}{2} \right)}{n}\)
\vspace{10in}
\end{enumerate}
\end{frame}
\begin{frame}[label={sec:orgd2f90ed}]{Alternating series test}
To show how we can determine the convergence or divergence of an
alternating series, let's look at a specific alternating series, the
\alert{alternating harmonic series}:
\vspace{10in}
\end{frame}
\begin{frame}[label={sec:org604452a}]{Alternating series test}
\end{frame}
\begin{frame}[label={sec:org8f2f4f5}]{Alternating series test}
\begin{theorem}[The Alternating Series Test]
An alternating series of the form
\[
\sum\limits_{n=1}^{\infty} \left( -1 \right)^n b_n \text{ or } \sum\limits_{n=1}^{\infty} \left( -1 \right)^{n+1}b_n\]
converges if:
\begin{enumerate}
\item 

\item 
\end{enumerate}
\end{theorem}
\end{frame}

\begin{frame}[label={sec:org4418213}]{Example}
Which of the following series converge:
\begin{enumerate}
\item \(\sum\limits_{n=1}^{\infty} \left( -1 \right)^n \frac{1}{n^2}\)
\item \(\sum\limits_{n=1}^{\infty} \left( -1 \right)^{n+1} \frac{1}{e^n}\)
\item \(\sum\limits_{n=1}^{\infty} \left( -1 \right)^n \ln(n)\)
\vspace{10in}
\end{enumerate}
\end{frame}

\begin{frame}[label={sec:orgfef8c6e}]{The Remainder of an Alternating Series}
Let's see if we can get a bound on the remainder \(R_N =
\sum\limits_{n=1}^{\infty} \left( -1 \right)^n b_n -
\sum\limits_{n=1}^{N} \left( -1 \right)^n b_n\) if the series converges.
\vspace{10in}
\end{frame}

\begin{frame}[label={sec:org677b8c8}]{Example}
What is the remainder if we use \[ \sum\limits_{n=0}^5 \left( -1
\right)^n \frac{1}{\left( 2n+1 \right)!}\] to approximate
\(\sum\limits_{n=0}^{\infty} \left( -1 \right)^n \frac{1}{\left( 2n+1
\right)!}\)?  (This is \(\sin(1)\), by the way).
\vspace{10in}
\end{frame}

\begin{frame}[label={sec:org2f43ec6}]{Absolute vs Conditional Convergence}
We've shown now that \(\sum\limits_{n=1}^{\infty} \frac{\left( -1 \right)^{n+1}}{n}\) converges,
but we also know that
\[
\sum\limits_{n=1}^{\infty} \left| \frac{\left( -1 \right)^{n+1}}{n} \right| = \sum\limits_{n=1}^{\infty} \frac{1}{n}\]
\alert{diverges} (why)?

A series \(\sum\limits_{n=1}^{\infty} a_n\) that converges, but for which
\(\sum\limits_{n=1}^{\infty} \left| a_n \right|\) diverges is called
\uline{\hspace*{1in}}.

A series \(\sum\limits_{n=1}^{\infty} a_n\) that converges, and for which
\(\sum\limits_{n=1}^{\infty} \left| a_n \right|\) converges is called
\uline{\hspace*{1in}}.
\end{frame}

\begin{frame}[label={sec:org9385a54}]{Example}
Is the series
\[
\sum\limits_{n=1}^{\infty} \left( \frac{-1}{4} \right)^n\]
conditionally convergent, absolutely convergent, or divergent?
\vspace{1in}
\end{frame}

\begin{frame}[label={sec:org349afaf}]{A useful theorem}
A fact that is useful is that if \(\sum\limits_{n=1}^{\infty} a_n\) is an absolutely
convergent series, then the original series \(\sum\limits_{n=1}^{\infty} a_n\) is also
convergent.
\vspace{10in}
\end{frame}
\end{document}