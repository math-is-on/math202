% Created 2024-01-23 Tue 09:26
% Intended LaTeX compiler: pdflatex
\documentclass[presentation]{beamer}
\usepackage[utf8]{inputenc}
\usepackage[T1]{fontenc}
\usepackage{graphicx}
\usepackage{longtable}
\usepackage{wrapfig}
\usepackage{rotating}
\usepackage[normalem]{ulem}
\usepackage{amsmath}
\usepackage{amssymb}
\usepackage{capt-of}
\usepackage{hyperref}
\mode<beamer>{\usetheme{Madrid}}
\definecolor{SUred}{rgb}{0.59375, 0, 0.17969} % SU red (primary)
\definecolor{SUblue}{rgb}{0, 0.17578, 0.38281} % SU blue (secondary)
\setbeamercolor{palette primary}{bg=SUred,fg=white}
\setbeamercolor{palette secondary}{bg=SUblue,fg=white}
\setbeamercolor{palette tertiary}{bg=SUblue,fg=white}
\setbeamercolor{palette quaternary}{bg=SUblue,fg=white}
\setbeamercolor{structure}{fg=SUblue} % itemize, enumerate, etc
\setbeamercolor{section in toc}{fg=SUblue} % TOC sections
% Override palette coloring with secondary
\setbeamercolor{subsection in head/foot}{bg=SUblue,fg=white}
\setbeamercolor{date in head/foot}{bg=SUblue,fg=white}
\institute[SU]{Shenandoah University}
\titlegraphic{\includegraphics[width=0.5\textwidth]{\string~/Documents/suLogo/suLogo.pdf}}
\newcommand{\R}{\mathbb{R}}
\usepackage{tikz}
\usepackage{pgfplots}
\usetheme{default}
\author{Chase Mathison\thanks{cmathiso@su.edu}}
\date{24 January 2024}
\title{\(u-\)substitution}
\hypersetup{
 pdfauthor={Chase Mathison},
 pdftitle={\(u-\)substitution},
 pdfkeywords={},
 pdfsubject={},
 pdfcreator={Emacs 29.1 (Org mode 9.6.7)}, 
 pdflang={English}}
\begin{document}

\maketitle

\section{Announcements}
\label{sec:orgb3dc0ad}
\begin{frame}[label={sec:org13bcc1c}]{Announcements}
\begin{enumerate}
\item Homework in MyOpenMath.
\item Office hours cancelled today.
\end{enumerate}
\end{frame}

\section{Lecture}
\label{sec:orgfd8fe94}
\begin{frame}[label={sec:orgc34ff36}]{\(u-\)substitution: a closer look}
Sometimes it is ``obvious'' that we need to use a \(u-\) substitution
when evaluating an integral, such as
\[
\int\limits_0^1 x \left( x^2-4 \right)^{10}\,dx \]

For this integral, it's pretty clear that if we make the \(u-\)substitution
\begin{align*}
u = & x^2-4 \\
\frac{1}{2}du = & x\,dx
\end{align*}
\end{frame}

\begin{frame}[label={sec:org39bddab}]{\(u-\)substitution: a closer look}
Then, this integral transforms into
\[
\int\limits_{-4}^{-3} \frac{1}{2}u^{10}\,du \]
which can be evaluated simply using the power rule:
\vspace{10in}
\end{frame}

\begin{frame}[label={sec:orgd85df03}]{\(u-\)substitution: a closer look}
\(u-\)substitutions can be used in other situations that
aren't quite as obvious, or as a preliminary step to make an integral
have a simpler form to use other techniques of integration.

For instance, let's try to use \(u-\)substitution to evaluate
\[
\int\limits_{}^{} \frac{x}{\sqrt{x-1}}\,dx \]
\vspace{10in}
\end{frame}

\begin{frame}[label={sec:org72437e2}]{\(u-\)substitution: a closer look}
\end{frame}

\begin{frame}[label={sec:org8d93611}]{\(u-\)substitution}
Here is a general problem solving strategy for integrals involving
\(u-\)substitution: 

\begin{enumerate}
\item Look at the integrand to determined if there is a composition of
functions of the form \(f \left( g \left( x \right) \right)\).
\item Substitute \(u = g \left( x \right)\) and \(du = g' \left( x
   \right)\,dx\).
\item If there are any \(x\)'s remaining in the integral after this
substitution, replace them using \(u = g(x)\).
\item Evaluate the integral in terms of \(u\), if possible.  If it is
not possible, we might need to go back and change our \(u-\)substitution.
\item Write your final answer in terms of \(x\) if finding an indefinite
integral.
\end{enumerate}
\end{frame}

\begin{frame}[label={sec:org8409de6}]{Examples}
Evaluate the following integral:
\[
\int\limits_{}^{} x \left( 1-x \right)^{99}\,dx \]
\vspace{10in}
\end{frame}


\begin{frame}[label={sec:orga6f2c38}]{Examples}
\end{frame}

\begin{frame}[label={sec:org5894ac5}]{Examples}
Evaluate the following integral:
\[
\int\limits_{-1}^{1}t \left( 1-t^2 \right)^{10}\,dt \]
\vspace{10in}
\end{frame}

\begin{frame}[label={sec:org12994b9}]{Examples}
\end{frame}

\begin{frame}[label={sec:org06aa1a3}]{Examples}
Evaluate the following integral:
\[
\int\limits_{}^{}x \sqrt{x+1}\,dx \]
\vspace{10in}
\end{frame}

\begin{frame}[label={sec:orgaf53421}]{Examples}
\end{frame}

\begin{frame}[label={sec:orgf53df1a}]{Examples}
Evaluate the following integral:
\[
\int \cos^3 \left(  \theta \right) \sin \left( \theta \right)\,d\theta
\]
\vspace{10in}
\end{frame}

\begin{frame}[label={sec:org576e128}]{Examples}
\end{frame}

\begin{frame}[label={sec:orgb6a885f}]{Examples}
Evaluate the following integral:
\[
\int\limits_0^{\pi/2} \cos^3 \left( \theta \right)\,d\theta \]
\vspace{10in}
\end{frame}

\begin{frame}[label={sec:orga6d6867}]{Examples}
\end{frame}

\begin{frame}[label={sec:org4ec73ac}]{Examples}
Evaluate the following integral:
\[
\int\limits_{}^{} t \sin \left( t^2 \right)\cos \left( t^2 \right)\,dt
\]
\vspace{10in}
\end{frame}
\end{document}