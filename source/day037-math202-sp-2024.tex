% Created 2024-04-09 Tue 21:16
% Intended LaTeX compiler: pdflatex
\documentclass[presentation]{beamer}
\usepackage[utf8]{inputenc}
\usepackage[T1]{fontenc}
\usepackage{graphicx}
\usepackage{longtable}
\usepackage{wrapfig}
\usepackage{rotating}
\usepackage[normalem]{ulem}
\usepackage{amsmath}
\usepackage{amssymb}
\usepackage{capt-of}
\usepackage{hyperref}
\mode<beamer>{\usetheme{Madrid}}
\definecolor{SUred}{rgb}{0.59375, 0, 0.17969} % SU red (primary)
\definecolor{SUblue}{rgb}{0, 0.17578, 0.38281} % SU blue (secondary)
\setbeamercolor{palette primary}{bg=SUred,fg=white}
\setbeamercolor{palette secondary}{bg=SUblue,fg=white}
\setbeamercolor{palette tertiary}{bg=SUblue,fg=white}
\setbeamercolor{palette quaternary}{bg=SUblue,fg=white}
\setbeamercolor{structure}{fg=SUblue} % itemize, enumerate, etc
\setbeamercolor{section in toc}{fg=SUblue} % TOC sections
% Override palette coloring with secondary
\setbeamercolor{subsection in head/foot}{bg=SUblue,fg=white}
\setbeamercolor{date in head/foot}{bg=SUblue,fg=white}
\institute[SU]{Shenandoah University}
\titlegraphic{\includegraphics[width=0.5\textwidth]{\string~/Documents/suLogo/suLogo.pdf}}
\newcommand{\R}{\mathbb{R}}
\usepackage{tikz}
\usepackage{pgfplots}
\usetheme{default}
\author{Chase Mathison\thanks{cmathiso@su.edu}}
\date{10 April 2024}
\title{The Integral Test}
\hypersetup{
 pdfauthor={Chase Mathison},
 pdftitle={The Integral Test},
 pdfkeywords={},
 pdfsubject={},
 pdfcreator={Emacs 29.1 (Org mode 9.6.7)}, 
 pdflang={English}}
\begin{document}

\maketitle

\section{Announcements}
\label{sec:org5f19695}
\begin{frame}[label={sec:orgfdbad9e}]{Announcements}
\begin{enumerate}
\item Homework in MyOpEnMaTh.
\item Office hours, 10am - 11am.
\end{enumerate}
\end{frame}

\section{Lecture}
\label{sec:org89f4f62}
\begin{frame}[label={sec:org37b8f9a}]{The Integral Test}
Let's look at another test that will bring back improper integrals!  Let's look at 2 specific examples to illustrate the \uline{\hspace*{1in}}.
\[
\sum\limits_{n=1}^{\infty} \frac{1}{n}\qquad \text{and} \quad \sum\limits_{n=1}^{\infty} \frac{1}{n^2}\]
\vspace{10in}
\end{frame}

\begin{frame}[label={sec:orgc51cad6}]{The Integral Test}
\end{frame}

\begin{frame}[label={sec:org7066ab1}]{The Integral Test}
\end{frame}

\begin{frame}[label={sec:orgfe3e3c0}]{The Integral Test}
What we just did works in general, as long as the series in question satisfies a few conditions.

\begin{theorem}[The Integral Test]
Suppose \(\sum\limits_{n=1}^{\infty} a_n\) is a series with \alert{strictly} \uline{\hspace*{1in}} \alert{terms}.  If there is a continuous function
\(f(x)\) and an integer \(N\) such that
\begin{enumerate}
\item \(f\) is a decreasing function, and
\item \(f(n) = a_n\) for all \(n \ge N\),
\end{enumerate}

Then
\[
\sum\limits_{n=1}^{\infty} a_n \quad \text{and} \quad \int\limits_N^{\infty} f(x)\,dx\]
\uline{\hspace*{3in}}.
\end{theorem}
\end{frame}

\begin{frame}[label={sec:orgecdf6ba}]{Example}
Determine which of the following series converge by using the integral test:
\begin{enumerate}
\item \[\sum\limits_{n=1}^{\infty} \frac{1}{n^2}\]
\item \[\sum\limits_{n=2}^{\infty} \frac{1}{n\ln(n)}\]
\item \[\sum\limits_{n=1}^{\infty} \frac{1}{n^2 + n}\]
\end{enumerate}
\vspace{10in}
\end{frame}

\begin{frame}[label={sec:org83c1679}]{Example}
\end{frame}

\begin{frame}[label={sec:orga4c0fa3}]{\(p-\)series}
Show that
\[
\sum\limits_{n=1}^{\infty} \frac{1}{n^p}\]
converges if \(p > 1\) and diverges if \(p \le 1.\)
\vspace{10in}
\end{frame}

\begin{frame}[label={sec:org6f629f2}]{\(p-\)series}
\end{frame}

\begin{frame}[label={sec:orgfd85a7f}]{Another use for the integral test}
In general if \(a_n = f(n)\) for a continuous, decreasing function,
and \(\sum\limits_{n=1}^{\infty} a_n\) converges, it is \alert{NOT} the case
that \[ \sum\limits_{n=1}^{\infty} a_n = \int\limits_1^{\infty}
f(x)\,dx\] But, we can still gain information about the value of the
series from the value of the improper integral.  Let's see how:
\vspace{10in}
\end{frame}

\begin{frame}[label={sec:org94d0d7a}]{Another use for the integral test}
\end{frame}

\begin{frame}[label={sec:org3adae89}]{Integral test remainder estimate}
We've shown the following:

Suppose \(\sum\limits_{n=1}^{\infty} a_n\) is a convergent series that satisfies
the criteria for use with the integral test (with corresponding function \(f\)).

Let \(S_N\) denote the partial sum \(\sum\limits_{n=1}^N a_n\). Then
\[
S_N + \int\limits_{N+1}^{\infty} f(x)\,dx < \sum\limits_{n=1}^{\infty}a_n < S_N + \int\limits_N^{\infty} f(x)\,dx. \]
If we denote \(R_N = \sum\limits_{n=1}^{\infty} a_n - S_N\) (called the \uline{\hspace*{1in}}), then another way to say this is
\[
\int\limits_{N+1}^{\infty} f(x)\,dx < R_N < \int\limits_N^{\infty} f(x)\,dx.\]
\end{frame}

\begin{frame}[label={sec:org37eeacb}]{Example}
It can be shown that
\[
\sum\limits_{n=1}^{\infty} \frac{1}{n^2} = \frac{\pi^2}{6}.\]
How many terms do we need to use in a partial sum to estimate this series
with a remainder (error) of \(10^{-3}\)?
\vspace{10in}
\end{frame}

\begin{frame}[label={sec:org07a6541}]{Example}
\end{frame}
\end{document}