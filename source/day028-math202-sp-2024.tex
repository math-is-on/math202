% Created 2024-03-18 Mon 12:09
% Intended LaTeX compiler: pdflatex
\documentclass[presentation]{beamer}
\usepackage[utf8]{inputenc}
\usepackage[T1]{fontenc}
\usepackage{graphicx}
\usepackage{longtable}
\usepackage{wrapfig}
\usepackage{rotating}
\usepackage[normalem]{ulem}
\usepackage{amsmath}
\usepackage{amssymb}
\usepackage{capt-of}
\usepackage{hyperref}
\mode<beamer>{\usetheme{Madrid}}
\definecolor{SUred}{rgb}{0.59375, 0, 0.17969} % SU red (primary)
\definecolor{SUblue}{rgb}{0, 0.17578, 0.38281} % SU blue (secondary)
\setbeamercolor{palette primary}{bg=SUred,fg=white}
\setbeamercolor{palette secondary}{bg=SUblue,fg=white}
\setbeamercolor{palette tertiary}{bg=SUblue,fg=white}
\setbeamercolor{palette quaternary}{bg=SUblue,fg=white}
\setbeamercolor{structure}{fg=SUblue} % itemize, enumerate, etc
\setbeamercolor{section in toc}{fg=SUblue} % TOC sections
% Override palette coloring with secondary
\setbeamercolor{subsection in head/foot}{bg=SUblue,fg=white}
\setbeamercolor{date in head/foot}{bg=SUblue,fg=white}
\institute[SU]{Shenandoah University}
\titlegraphic{\includegraphics[width=0.5\textwidth]{\string~/Documents/suLogo/suLogo.pdf}}
\usepackage{tikz}
\usetheme{default}
\author{Chase Mathison\thanks{cmathiso@su.edu}}
\date{19 March 2024}
\title{Mass and Moments}
\hypersetup{
 pdfauthor={Chase Mathison},
 pdftitle={Mass and Moments},
 pdfkeywords={},
 pdfsubject={},
 pdfcreator={Emacs 29.1 (Org mode 9.6.7)}, 
 pdflang={English}}
\begin{document}

\maketitle

\section{Announcements}
\label{sec:orge0dfb24}
\begin{frame}[label={sec:org1be8bd4}]{Announcements}
\begin{enumerate}
\item Exam next Monday.
\item Office hours, 10am - 11am.
\end{enumerate}
\end{frame}

\section{Mass}
\label{sec:orgb7507e5}
\begin{frame}[label={sec:orgee8ac2e}]{Center of mass and moment of masses on a line}
We'll start with masses on a line.  If we have 2 (not necessarily
equal) masses on a line (think of a see-saw) then what would be the
point where we could perfectly balance that line?  This is called the
\uline{\hspace*{1in}}. Let's find this: 
\vspace{10in}
\end{frame}

\begin{frame}[label={sec:orgb0eea70}]{Center of mass}
In general, if we have \(n\) point masses at \(n\) points, we have the
following:
\begin{theorem}[Center of mass]
If \(m_1, m_2,\ldots m_n\) are masses at the points \(x_1,x_2,\ldots, x_n\) respectively, then we have the center of mass
\[
\bar{x} = \hspace{2in}\]

The value \(M = m_1x_1 + m_2x_2\) is called the \uline{\hspace*{1in}} of the system.
\end{theorem}
\end{frame}

\begin{frame}[label={sec:org1b81698}]{Example}
Find the center of mass and the moment of the system with respect to
the origin for the system of masses:
\begin{align*}
m_1 = 2 kg \text{ at } x_1 = 0.2 m, \,\,& m_2 = 3 kg \text{ at } x_2 = 1 m, \\
m_3 = 7 kg \text{ at } x_3 = -0.5 m, \,\,& m_4 = 3.5 kg \text{ at } x_4 = -0.75 m
\end{align*}
\vspace{10in}
\end{frame}

\begin{frame}[label={sec:org80b9bab}]{Example}
\end{frame}

\begin{frame}[label={sec:orgcfe5098}]{Center of mass and moments of system in the plane}
We can generalize the previous ideas to masses in the \(xy-\)plane.
Let's do this for 2 masses:
\vspace{10in}
\end{frame}

\begin{frame}[label={sec:orgbf89fb1}]{Center of mass and moments of system in the plane}
Again, we can generalize this to \(n\) masses in the plane at \(n\)
points:
\begin{theorem}[Center of mass and moments in the plane]
If \(m_1,m_2,\ldots,m_n\) are point masses at the points \(\left(
x_1,y_1 \right), \left( x_2,y_2 \right),\ldots \left( x_n,y_n
\right)\) respectively, then we define the quantities
\[
M_x = \hspace{2in}\]
\[
M_y = \hspace{2in}\]
\[
\bar{x} = \hspace{2in}\]
\[
\bar{y} = \hspace{2in}\]
\end{theorem}

\vspace{10in}
\end{frame}

\begin{frame}[label={sec:orgcfbc57f}]{Example}
Find the moments \(M_x,M_y\) and the center of mass of
the system of masses:
\begin{align*}
m_1 = 2 kg,& \text{ at } \left( -1,3 \right) \\
m_2 = 6 kg,& \text{ at } \left( 1,1 \right) \\
m_3 = 4 kg,& \text{ at } \left( 2,-2 \right)
\end{align*}
\vspace{10in}
\end{frame}

\begin{frame}[label={sec:orga4b23c0}]{Center of mass of a thin sheet}
But what if we want to find the center of mass (or \alert{centroid}, which
is the geometric center of an object) of a 2 dimensional
object, like a thin plate or some other thin object defined by a
function?

For what follows, we're going to find the moments about the \(x\) and
\(y\) axis and the center of mass for what's called a \emph{lamina}, which
is a thin sheet of uniform density \(\rho\) represented as a region
in the \(xy-\)plane. 
\vspace{10in}
\end{frame}

\begin{frame}[label={sec:org0be95b2}]{Center of mass of a thin sheet}
Can you guess what we're about to do?
\vspace{10in}
\end{frame}

\begin{frame}[label={sec:orgaec5de9}]{Center of mass of a thin sheet}
\end{frame}

\begin{frame}[label={sec:org6551e07}]{Center of mass of a thin sheet}
\begin{theorem}[Center of mass of a thin sheet]
Suppose \(R\) is the region bounded above by the continous function
\(y = f(x)\), below by \(y = 0\) and on the left and right by \(x =
a\) and \(x = b\) respectively. Let \(\rho\) (rho) denote the density of the associated lamina.  Then we have the following:

\begin{enumerate}
\item \(m = \hspace{2in}\)
\item \[M_x = \hspace{2in}\] and \[M_y = \hspace{2in}\]
\item \[\bar{x} = \hspace{2in}\] and \[\bar{y} = \hspace{2in}\]
\end{enumerate}
\end{theorem}

\vspace{10in}
\end{frame}

\begin{frame}[label={sec:org00a1ff9}]{Example}
Let \(R\) be the region bounded by the curve \(y=\sqrt{x}\), the
\(x-\)axis and \(x=4\).  This region has a constant density of \(2\)kg/m\(^2\) where \(x\) is measured in meters.  Find the center of mass of
this region (lamina).  What is the centroid?
\vspace{10in}
\end{frame}

\begin{frame}[label={sec:orga1e1c79}]{Example}
\end{frame}

\begin{frame}[label={sec:org5a99fa1}]{More general lamina}
Let's look at what happens if the lamina we are examining is defined
by an upper curve \(y = f(x)\) and a lower curve \(y = g(x),\) again
with a constant density \(\rho\) which has units of mass per
length\(^2\).
\vspace{10in}
\end{frame}

\begin{frame}[label={sec:orgd8f6273}]{More general lamina}
In general, assume \(R\) is the region bounded above by the graph of
\(y = f(x)\), below by the graph of \(y = g(x)\) and on the left and
right by \(x = a\) and \(x = b\) respectively.  Also, suppose that the density of the associated lamina is the constant \(\rho.\) Then:

\vspace{10in}
\end{frame}

\begin{frame}[label={sec:org3da8c46}]{Example}
Find the center of mass of the region bounded by the curves
\(y = 1-x^2\) and \(y = x\), with a constant density of \(\rho=1\)kg/m\textsuperscript{2} where \(x\) is in m.
\vspace{10in}
\end{frame}

\begin{frame}[label={sec:orgf0b9ee9}]{Example}
\end{frame}

\begin{frame}[label={sec:org9f7b6f9}]{Example (The symmetry principal)}
Find the center of mass of the region bounded by the curves \(y =
x^2-1\) and \(\sqrt{1-x^2}\), with constant density of \(\rho=3\)kg/m\textsuperscript{2} where \(x\) is in m.
\vspace{10in}
\end{frame}

\begin{frame}[label={sec:org05ec2b9}]{Example}
\end{frame}

\begin{frame}[label={sec:org291dc6c}]{One more example}
Find the center of mass of the region bounded by the curve \(y =
\cos(x)\) and the \(x-\)axis between \(x=-\pi/2\) and \(x=\pi/2\),
where the lamina has a constant density of \(1\)kg/m\(^2\) where \(x\) is in meters.
\vspace{10in}
\end{frame}
\end{document}