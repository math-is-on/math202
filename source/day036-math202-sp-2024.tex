% Created 2024-04-09 Tue 21:04
% Intended LaTeX compiler: pdflatex
\documentclass[presentation]{beamer}
\usepackage[utf8]{inputenc}
\usepackage[T1]{fontenc}
\usepackage{graphicx}
\usepackage{longtable}
\usepackage{wrapfig}
\usepackage{rotating}
\usepackage[normalem]{ulem}
\usepackage{amsmath}
\usepackage{amssymb}
\usepackage{capt-of}
\usepackage{hyperref}
\mode<beamer>{\usetheme{Madrid}}
\definecolor{SUred}{rgb}{0.59375, 0, 0.17969} % SU red (primary)
\definecolor{SUblue}{rgb}{0, 0.17578, 0.38281} % SU blue (secondary)
\setbeamercolor{palette primary}{bg=SUred,fg=white}
\setbeamercolor{palette secondary}{bg=SUblue,fg=white}
\setbeamercolor{palette tertiary}{bg=SUblue,fg=white}
\setbeamercolor{palette quaternary}{bg=SUblue,fg=white}
\setbeamercolor{structure}{fg=SUblue} % itemize, enumerate, etc
\setbeamercolor{section in toc}{fg=SUblue} % TOC sections
% Override palette coloring with secondary
\setbeamercolor{subsection in head/foot}{bg=SUblue,fg=white}
\setbeamercolor{date in head/foot}{bg=SUblue,fg=white}
\institute[SU]{Shenandoah University}
\titlegraphic{\includegraphics[width=0.5\textwidth]{\string~/Documents/suLogo/suLogo.pdf}}
\newcommand{\R}{\mathbb{R}}
\usepackage{tikz}
\usepackage{pgfplots}
\usetheme{default}
\author{Chase Mathison\thanks{cmathiso@su.edu}}
\date{9 April 2024}
\title{The Geometric Series and the Divergence Test}
\hypersetup{
 pdfauthor={Chase Mathison},
 pdftitle={The Geometric Series and the Divergence Test},
 pdfkeywords={},
 pdfsubject={},
 pdfcreator={Emacs 29.1 (Org mode 9.6.7)}, 
 pdflang={English}}
\begin{document}

\maketitle

\section{Announcements}
\label{sec:org5da53c4}
\begin{enumerate}
\item Homework in MyOpenMath
\item Office hours, 10am - 11am
\item Free coffee and cookies on Fridays in MEC at noon.
\end{enumerate}

\section{Lecture}
\label{sec:orgc8a3afc}
\begin{frame}[label={sec:org43f779e}]{The Geometric Series}
Now let's investigate a very special type of a series: a series in which the individual terms
form a geometric sequence will be called a \uline{\hspace*{1in}} and has the form
\[
\sum\limits_{n=0}^{\infty} ar^n = a + ar + ar^2 + ar^3 + \ldots\]
Let's see what we can say about a series like this.

\vspace{10in}
\end{frame}

\begin{frame}[label={sec:orgf062e3f}]{The Geometric Series}
\end{frame}

\begin{frame}[label={sec:org6053274}]{The Geometric Series}
We've shown

\begin{theorem}[Geometric Series]
The series
\[
a + ar + ar^2 + ar^3 + \ldots = \sum\limits_{n=0}^{\infty} ar^n\]
\uline{\hspace*{2in}} if \(|r| < 1\) and \uline{\hspace*{1in}} if \(|r| \ge 1.\)
\end{theorem}
\end{frame}

\begin{frame}[label={sec:org0684c8f}]{Example}
What is
\[
.123123123\ldots\]
as a fraction?
\vspace{10in}
\end{frame}

\begin{frame}[label={sec:orgbdfb22b}]{Example}
\end{frame}

\begin{frame}[label={sec:orgd4a7208}]{Example}
What is the area of the Sierpinski Triangle?
\vspace{10in}
\end{frame}

\begin{frame}[label={sec:orga22d791}]{Example}
\end{frame}

\begin{frame}[label={sec:orgae498f6}]{The Divergence Test}
It would be nice if, given a series, there was a quick way to tell if the series
was divergent.  Thankfully, we have the divergence test for that!

\begin{theorem}[The Divergence Test]
If \(\lim_{n\rightarrow \infty} a_n \neq 0\), then the series
\[
\sum\limits_{n=1}^{\infty} a_n\]
\uline{\hspace*{1in}}.
\end{theorem}

Note!
\end{frame}

\begin{frame}[label={sec:org63164ab}]{Example}
Which of the following series can we immediately say diverges?
\begin{enumerate}
\item \[\sum\limits_{n=1}^{\infty} \frac{1}{n!}\]
\item \[\sum\limits_{n=2}^{\infty} \frac{n+1}{n-1}\]
\item \[\sum\limits_{n=1}^{\infty} n^{\frac{1}{n}}\]
\end{enumerate}
\vspace{10in}
\end{frame}

\begin{frame}[label={sec:org45d6227}]{Example}
\end{frame}
\end{document}